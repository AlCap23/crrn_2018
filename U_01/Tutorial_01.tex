\hypertarget{tutorial-1---basic-multibody-simulation}{%
\section{Tutorial 1 - Basic Multibody
Simulation}\label{tutorial-1---basic-multibody-simulation}}

\hypertarget{abstract}{%
\subsection{Abstract}\label{abstract}}

In this practical lecture you should gather (first) experience with a
multibody simulation. For an easy start, we use
\href{https://github.com/bulletphysics/bullet3}{Bullet Physics} with its
natural binding \href{https://pybullet.org/wordpress/}{PyBullet}.
Recently, PyBullet has been used in different scientific applications,
e.g. \href{https://youtu.be/t-gSFn24ZKs}{deep-learning for mimic
movement}, \href{https://youtu.be/xf_UXK0OTIk}{simulation-reality gap
learning} and others.

The Bullet Physics Library is still under development and incorporates
many usefull features like inverse kinematics, control, ray tracing.

In this tutorial you should: - Get to know some basic simulation
parameters - See the influence of geometry on the simulation - Create a
simple multibody via scripting

\hypertarget{a-word-of-advice}{%
\subsection{A word of Advice}\label{a-word-of-advice}}

\begin{itemize}
\item
  If you feel stuck, feel free to ask. However, the goal is that you
  play with the simulation and get a feeling for the influence of
  different parameters.
\item
  Pybullet has a decent API documentation, the
  \href{https://docs.google.com/document/d/10sXEhzFRSnvFcl3XxNGhnD4N2SedqwdAvK3dsihxVUA/edit\#heading=h.2ye70wns7io3}{Pybullet
  Quickstart Guide}, where almost every command is explained.
\item
  Futhermore, the
  \href{https://github.com/bulletphysics/bullet3/tree/master/examples/pybullet/examples}{Examples
  Folder} within the Github Repository is full of \ldots{}examples.
\end{itemize}

We will use \href{https://github.com/jupyterlab/jupyterlab}{Jupyter Lab}
for this tutorial, which enables an interactive use of python - and
other languages. You can start via:

\begin{Shaded}
\begin{Highlighting}[]
\CommentTok{# Use these commands in a terminal}

\CommentTok{# Activate the simulation environment of conda}
\BuiltInTok{source}\NormalTok{ activate crrn_sim}

\CommentTok{# Navigate to an appropriate folder of your choice}
\BuiltInTok{cd}\NormalTok{ my/simulation/folder}

\CommentTok{# Start a jupyter lab server}
\ExtensionTok{jupyter}\NormalTok{ lab}
\end{Highlighting}
\end{Shaded}

A browser opens up, where you will find a user interface. Select
\texttt{File\ -\textgreater{}\ New\ -\textgreater{}\ Notebook} and
Python 3 as a kernel. Rename your notebook accordingly.

\hypertarget{part-1---simple-physics---simple-models}{%
\subsection{Part 1 - Simple Physics - Simple
models}\label{part-1---simple-physics---simple-models}}

We will start by investigating the classical examples which lead to the
model of gravity as given by Isaac Newton : A falling object. Oppose to
using an apple, a sphere will suffice for our investigation.

You need the following commands:

\begin{Shaded}
\begin{Highlighting}[]
\ImportTok{import}\NormalTok{ pybullet}
\ImportTok{import}\NormalTok{ pybullet_data}

\CommentTok{# Start a server with a graphical user interface}
\NormalTok{pybullet.}\ExtensionTok{connect}\NormalTok{(}\OperatorTok{*}\NormalTok{args)}

\CommentTok{# Get some objects to import}
\NormalTok{pybullet_data.getDataPath()}

\CommentTok{# Load a urdf}
\NormalTok{pybullet.loadURDF( filename, }\OperatorTok{*}\NormalTok{args)}

\CommentTok{# Set the gravity}
\NormalTok{pybullet.setGravity(}\OperatorTok{*}\NormalTok{args)}

\CommentTok{# And some way to step the simulation}
\NormalTok{pybullet.setRealTimeSimulation(}\OperatorTok{*}\NormalTok{args)}
\NormalTok{pybullet.stepSimulation(}\OperatorTok{*}\NormalTok{args)}
\end{Highlighting}
\end{Shaded}

\hypertarget{tasklist}{%
\subsubsection{Tasklist}\label{tasklist}}

\begin{itemize}
\tightlist
\item
  Start a simulation server
\item
  Get the data path of pybullet\_data
\item
  Add the objects `plane.urdf' and `sphere2.urdf' and load them into the
  simulation
\item
  Set the gravity accordingly
\item
  Simulate for a sufficient time period, e.g.~10 seconds
\end{itemize}

What do you see?

\hypertarget{hints}{%
\subsubsection{Hints}\label{hints}}

\begin{itemize}
\tightlist
\item
  Ideally, the sphere has to be placed above the plane.
\item
  Look up all commands in the
  \href{https://docs.google.com/document/d/10sXEhzFRSnvFcl3XxNGhnD4N2SedqwdAvK3dsihxVUA/edit\#heading=h.2ye70wns7io3}{Pybullet
  Quickstart Guide} to get the necessary \texttt{*args}
\end{itemize}

\hypertarget{notes}{%
\subsubsection{Notes}\label{notes}}

\pagebreak

\hypertarget{part-2---simple-physics---extended}{%
\subsection{Part 2 - Simple Physics -
Extended}\label{part-2---simple-physics---extended}}

Next, we want to enhance the simulation. We clearly need to change some
settings to get a real behaviour.

First, remember that the momentum of two colliding objects is (ideally)
conserved

\(p_1(t_0)+p_2(t_0)=p_1(t_1)+p_2(t_1)\)

However, due to thermodynamics, we can assume that the overall energy of
the mechanical system is constant or decreasing.

\(E(t_0) = \sum_i \frac{1}{2 m_i} p_i^2(t_0)\geq E(t_1)\)

Which is modeled via the restitution of a collision

\(p_i(t_1) = m_i v_i (t_1) = e ~ m_i v_i(t_0)\)

We can manipulate these parameters in Pybullet as well. Try the
following commands:

\begin{Shaded}
\begin{Highlighting}[]
\CommentTok{# Get available information on a body}
\NormalTok{pybullet.getDynamicsInfo(}\OperatorTok{*}\NormalTok{args)}

\CommentTok{# Change that information}
\NormalTok{pybullet.changeDynamics(}\OperatorTok{*}\NormalTok{args)}

\CommentTok{# Can you make a debug parameter to change the dynamics on the fly}
\NormalTok{pybullet.addUserDebugParameter(}\OperatorTok{*}\NormalTok{args)}
\NormalTok{pybullet.readUserDebugParameter(}\OperatorTok{*}\NormalTok{args)}
\end{Highlighting}
\end{Shaded}

\hypertarget{hints-1}{%
\subsubsection{Hints}\label{hints-1}}

\begin{itemize}
\tightlist
\item
  If you use a simulation loop, you can apply the
  \texttt{pybullet.readUserDebugParameter} and
  \texttt{pybullet.changeDynamics(*args)} each loop
\item
  You can use this tactic with the real time simulation as well.
\end{itemize}

\hypertarget{notes-1}{%
\subsubsection{Notes}\label{notes-1}}

\pagebreak

\hypertarget{part-3---simple-physics---hyperparameter}{%
\subsection{Part 3 - Simple Physics -
Hyperparameter}\label{part-3---simple-physics---hyperparameter}}

Still, some influences remain which we have not touched yet. To explore
the hyperparameter of the simulation, we can use the following commands:

\begin{Shaded}
\begin{Highlighting}[]
\CommentTok{# Get the available parameter of the engine}
\NormalTok{pybullet.getPhysicsEngineParameter(}\OperatorTok{*}\NormalTok{args)}

\CommentTok{# Set them}
\NormalTok{pybullet.setPhysicsEngineParameter(}\OperatorTok{*}\NormalTok{args)}
\end{Highlighting}
\end{Shaded}

Where parameter like step size can be varied. Add user debug parameter
for \texttt{fixedTimeStep}, \texttt{numSubSteps}, \texttt{erp},
\texttt{contactERP} and \texttt{frictionERP} and see whats happening.

\hypertarget{notes-2}{%
\subsubsection{Notes}\label{notes-2}}

\pagebreak

\hypertarget{part-4---simple-physics---create-a-body}{%
\subsection{Part 4 - Simple Physics - Create a
Body}\label{part-4---simple-physics---create-a-body}}

Next, we will create another sphere with the API.

Copy the following code snippet into a jupyter cell and execute:

\begin{Shaded}
\begin{Highlighting}[]

\NormalTok{pybullet.resetSimulation()}

\NormalTok{pybullet.loadURDF(pybullet_data.getDataPath() }\OperatorTok{+} \StringTok{'/plane.urdf'}\NormalTok{, useFixedBase }\OperatorTok{=} \VariableTok{True}\NormalTok{)}

\NormalTok{collision }\OperatorTok{=}\NormalTok{ pybullet.createCollisionShape(}
\NormalTok{    shapeType }\OperatorTok{=}\NormalTok{ pybullet.GEOM_BOX,}
\NormalTok{    halfExtents }\OperatorTok{=}\NormalTok{ [}\FloatTok{0.5}\NormalTok{, }\FloatTok{0.5}\NormalTok{, }\FloatTok{0.5}\NormalTok{]}
\NormalTok{)}

\NormalTok{visual }\OperatorTok{=}\NormalTok{ pybullet.createVisualShape(}
\NormalTok{    shapeType }\OperatorTok{=}\NormalTok{ pybullet.GEOM_SPHERE,}
\NormalTok{    radius }\OperatorTok{=} \FloatTok{0.5}
\NormalTok{)}

\NormalTok{new_sphere }\OperatorTok{=}\NormalTok{ pybullet.createMultiBody(}
\NormalTok{    baseMass }\OperatorTok{=} \FloatTok{1.0}\NormalTok{,}
\NormalTok{    baseCollisionShapeIndex }\OperatorTok{=}\NormalTok{ collision,}
\NormalTok{    baseVisualShapeIndex }\OperatorTok{=}\NormalTok{ visual,}
\NormalTok{    basePosition }\OperatorTok{=}\NormalTok{ [}\DecValTok{1}\NormalTok{, }\DecValTok{1}\NormalTok{, }\FloatTok{0.5}\NormalTok{]}
\NormalTok{)}
\end{Highlighting}
\end{Shaded}

A new sphere is created at position \(r = (1,1,1)\).

Create another sphere \texttt{new\_sphere\_2} at location
\(r=(1.25, 0.75, 1.8)\) and look what happens. Remember to set the
gravity and start the simulation.

\hypertarget{notes-3}{%
\subsubsection{Notes}\label{notes-3}}
